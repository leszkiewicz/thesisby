%\renewcommand{\mybibname}{}

\begin{center}{\textbf{СПИСОК ПУБЛИКАЦИЙ СОИСКАТЕЛЯ}}\end{center}
\addcontentsline{toc}{section}{Список публикаций соискателя}%

\renewcommand{\mybibname}{\normalsize \mbox{Статьи} в \mbox{изданиях,} \mbox{включенных} в \mbox{перечни} \mbox{научных} \mbox{изданий} \mbox{для опубликования} \mbox{результатов} \mbox{диссертационных} \mbox{исследований}}

\begin{myabstractbibliography}{10}
\def\selectlanguageifdefined#1{
\expandafter\ifx\csname date#1\endcsname\relax
\else\language\csname l@#1\endcsname\fi}
\ifx\undefined\url\def\url#1{{\small #1}}\else\fi
\ifx\undefined\BibUrl\def\BibUrl#1{\url{#1}}\else\fi
\ifx\undefined\BibAnnote\long\def\BibAnnote#1{}\else\fi
\ifx\undefined\BibEmph\def\BibEmph#1{\emph{#1}}\else\fi


% #1
\bibitem{myarticle1}
\selectlanguageifdefined{russian}
Название~статьи 1 / ~А.~А.~Автор1, ~А.~А.~Автор2, ~А.~А.~Автор3. ~А.~А.~Автор4~// ~Название~ ~Журнала. "---
\newblock 2006. "---
\newblock {\cyr\textnumero}~8. "---\newblock {\cyr\CYRS.}~5--9.


%#2
\bibitem{myarticle2}
\selectlanguageifdefined{russian}
Название~статьи 2 / ~А.~А.~Автор1, ~А.~А.~Автор2, ~А.~А.~Автор3, ~А.~А.~Автор4~// ~Название~ ~Журнала. "---
\newblock 2006. "---
\newblock {\cyr\textnumero}~8. "---\newblock {\cyr\CYRS.}~5--9.



\end{myabstractbibliography}


\renewcommand{\mybibname}{\normalsize \mbox{Статьи в научно-технических журналах}}

\begin{myabstractbibliography}{10}
\def\selectlanguageifdefined#1{
\expandafter\ifx\csname date#1\endcsname\relax
\else\language\csname l@#1\endcsname\fi}
\ifx\undefined\url\def\url#1{{\small #1}}\else\fi
\ifx\undefined\BibUrl\def\BibUrl#1{\url{#1}}\else\fi
\ifx\undefined\BibAnnote\long\def\BibAnnote#1{}\else\fi
\ifx\undefined\BibEmph\def\BibEmph#1{\emph{#1}}\else\fi

\setcounter{enumiv}{2} %тут вставить номер по порядку публикации из предыдущего раздела (здесь это номер 2  (см. выше %#2))

% #3
%эта НЕ ВАК
\bibitem{myarticle3}
\selectlanguageifdefined{russian}
Автор1,~А.~А. Влияние процессов ...~/ А.~А.~Автор1, А.~А.~Автор2~// 
  Журнал. "---
\newblock 2007. "---
\newblock {\cyr\textnumero} 7(40). "---
\newblock {\cyr\CYRS.}~58--61.

% #4
%эта НЕ ВАК
\bibitem{myarticle4}
\selectlanguageifdefined{russian}
Автор1,~А.~А. Влияние процессов ...~/ А.~А.~Автор1, А.~А.~Автор2, А.~А.~Автор3~// 
  Журнал. "---
\newblock 2013. "---
\newblock {\cyr\textnumero}~5. "---
\newblock {\cyr\CYRS.}~132--133.

\end{myabstractbibliography}



\renewcommand{\mybibname}{\normalsize \mbox{Материалы} \mbox{докладов} на \mbox{конференциях,} \mbox{семинарах,} \mbox{тезисы} \mbox{докладов}}

\begin{myabstractbibliography}{10}
\def\selectlanguageifdefined#1{
\expandafter\ifx\csname date#1\endcsname\relax
\else\language\csname l@#1\endcsname\fi}
\ifx\undefined\url\def\url#1{{\small #1}}\else\fi
\ifx\undefined\BibUrl\def\BibUrl#1{\url{#1}}\else\fi
\ifx\undefined\BibAnnote\long\def\BibAnnote#1{}\else\fi
\ifx\undefined\BibEmph\def\BibEmph#1{\emph{#1}}\else\fi

\setcounter{enumiv}{4} %тут вставить номер по порядку публикации из предыдущего раздела (здесь это номер 4)

%конференции

% #5
\bibitem{myarticle5}
\selectlanguageifdefined{russian}
Название ~/ А.~А.~Автор1, А.~А.~Автор2, А.~А.~Автор3,
  А.~А.~Автор4~// Наука -- образованию, производству, экономике: материалы
  Десятой междунар. науч.-техн. конф., Минск, 18–19 апр. 2006~г.: в~2~т.~/ Белорус. нац. техн. ун-т;
редкол.: И.~И.~Иванов, С.~С.~Сидоров, З.~З.~Забывайко. "---
\newblock \CYRT.~1. "---
\newblock Минск, 2006. "---
\newblock {\cyr\CYRS.}~11--19.

% #6
\bibitem{myarticle6}
\selectlanguageifdefined{russian}
Автор1,~А.~А. Название статьи~/ А.~А.~Автор1, А.~А.~Автор2~//
  Наука -- образованию, производству, экономике: материалы Пятой междунар. науч.-техн. конф., Минск, 11–12 апр. 2006~г.: в~12~т.~/ Белорус. нац. техн. ун-т;
редкол.: И.~И.~Иванов, С.~С.~Сидоров, З.~З.~Забывайко. "---
\newblock \CYRT.~1. "---
\newblock Минск, 2006. "---
\newblock {\cyr\CYRS.}~1--3.

\end{myabstractbibliography}
