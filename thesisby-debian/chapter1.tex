\chapter{ОБЗОРНАЯ}  \label{chapter1}


%%%надпись из [] идёт в содержание, надпись из {} идет заголовком в текст
\section[Очень очень длинное название первого параграфа]{Очень очень длинное название первого\\параграфа}\label{ch1-1}

Пример ссылки на литературу

@Article: \cite{PhysRevLett.116.167802, Polanyi1916, ermolenko1960}.

@Book: \cite{stallman2015free, trechsel2001}.

@Book: \cite{deriagin1984plenki, Lykov_stf}.

@Conference: \cite{capener2014art, art_krutilin_sorbcia2016}

@Electronic: \cite{bell2011pyamg, pogodaBY, vanovschi2015parallel}

@InCollection: \cite{Kuenzel_et_al_2001, bogoslovcky1966art}

@PhdThesis: \cite{descamps1996, kozlov_ktn}



@Manual (тут пишем всё что надо в поле Title): \cite{gost15588_2014}





\section[Еще один очень очень очень очень очень очень длинный заголовок]{Еще один очень очень очень очень очень очень\\длинный заголовок}\label{ch1-2}



Пример формулы:

\begin{equation}\label{eq:par_fokina}
\frac{\partial e}{\partial z} = \frac{\mu}{\xi_{o} \gamma}  E_{t} \frac{\partial^2 e}{\partial x^2},
\end{equation}

\begin{eqrem}
$x$ -- пространственная координата, м;\\
& $z$ -- временн\'{а}я координата, ч;\\
& $\mu$ -- коэффициент паропроницаемости материала, г/(м$\cdot$ч$\cdot$мм.рт.ст.);\\
& $\xi_{o}$ -- удельная относительная пароёмкость, г/кг;\\
& $\gamma$ -- объёмный вес материала, кг/м$^3$;\\
& $e$ -- упругость водяного пара, мм.рт.ст.;\\
& $E_{t}$ -- максимальная упругость водяного пара при температуре $t$, мм.рт.ст.
\end{eqrem}


Ещё один пример формулы:


\begin{equation}\label{eq:lukianov_t}
\begin{split}
\left( c_{o} \rho_{o} + c_{\text{ж}} u_{2} + c_{\text{л}} u_{3} + \chi r_{2} \frac{\beta_{3} u - b_{3}}{t^2} \right) \frac{\partial t}{\partial \tau} = \text{div}  \left[ \lambda (u,t) \nabla t + \rho_{\text{воз}} \frac{k_{\text{г}}(u)}{\mu_{\text{воз}}} c_{p} t \nabla P_{o} \right] + \\
+ r_{1} \cdot \text{div} \left[ \rho_{\text{воз}} \frac{M_{\text{в}}}{M_{\text{воз}}} \left( D_{\text{п}} (u)\nabla \frac{\varphi(u, c^*_{i}) \cdot E(t)}{P_{o}} + \frac{k_{\text{г}}(u)}{\mu_{\text{воз}}} \cdot \frac{\varphi(u, c^*_{i}) \cdot E(t)}{P_{o}} \nabla P_{o}\right) \right] +\\
+ \chi \cdot r_{2} \cdot (1-\alpha_{3} + \beta_{3}/t) \cdot \partial u / \partial \tau;\hspace{4cm}
\end{split}
\end{equation}

\begin{equation}\label{eq:lukianov_u}
\begin{split}
\frac{\partial u}{\partial \tau} = \text{div} \left[  \rho_{\text{воз}} \frac{M_{\text{в}}}{M_{\text{воз}}} \left( D_{\text{п}} (u)\nabla \frac{\varphi(u, c^*_{i}) \cdot E(t)}{P_{o}} + \frac{k_{\text{г}}(u)}{\mu_{\text{воз}}} \cdot \frac{\varphi(u, c^*_{i}) \cdot E(t)}{P_{o}} \nabla P_{o} \right)   \right] + \\
+ \text{div} \left[ K(u_{2}, t, c^*_{i}, I) \nabla u_{2} + K_{t}(u_{2}, t, c^*_{i}, I) \nabla t + K_{c}(u_{2}, t, I) \nabla c^*_{i} \right],\hspace{1.5cm}
\end{split}
\end{equation}

\begin{eqrem}
$c^*_{i}$ -- концентрация $i$-го компонента моли в поровом растворе;\\
& $u, u_{2}, u_{3}$ -- влагосодержание, кг/м$^3$, соответственно полное, в жидкой и твёрдой фазах;\\
& $t$ -- локальная температура,$^{\circ}$C;\\
& $c_{o}, c_{\text{ж}}, c_{\text{л}}, c_{p}$ -- теплоёмкость, соответственно сухого материала, воды, льда и воздуха, кДж/(кг$\cdot^{\circ}$C);\\
& $K, K_{t}, K_{c}$ -- соответственно коэффициенты влагопроводности (м$^2$/ч), термовлагопроводности (кг/(м$\cdot$ч$\cdot^{\circ}$C)), солепроводности (кг/(м$\cdot$ч$\cdot$н/л));\\               
& $D_{\text{п}}$ -- коэффициент диффузии водяного пара, м$^2$/ч;\\
& $k_{\text{г}}$ -- проницаемость среды по газу, м$^2$;\\
& $I$ -- напорный градиент;\\
& $\rho_{o}, \rho_{\text{возд}}$ -- плотность материала и воздуха, кг/м$^3$;\\
& $P_{o}$ -- давление воздуха, Па;\\
& $M_{\text{в}}, M_{\text{возд}}$ -- масса киломоля воды и воздуха, кг/кмоль;\\
& $E$ -- парциальное давление насыщенного водяного пара, Па;\\
& $\mu_{\text{воз}}$ -- вязкость воздуха, Па$\cdot$c;\\
& $r_{1}$ -- теплота фазового перехода вода-пар, кДж/кг;\\
& $r_{2}$ -- теплота фазового перехода вода-лёд, кДж/кг;\\
& $\alpha_{3}, \beta_{3}, b_{3}$ -- эмпирические коэффициенты.\\
\end{eqrem}





\section*{Выводы и задачи исследований}
\addcontentsline{toc}{section}{Выводы и задачи исследований}


Вот тут можно приклеить те же самые задачи исследований из характеристики работы...


--~задачи исследования можно подгрузить из отдельного файла (прямо этого);

--~и его же можно подгрузить где-нибудь в литобзоре когда ставится задачи исследований;

--~хотя позже мне посоветовали текст задач изменить, чтобы не было дословного копирования из литобзора;

--~хотя, кому оно мешает в двух местах -- я так и не понял;

--~спишем на бюрократию.

Либо сформулировать их другими словами...