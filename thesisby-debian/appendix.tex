\chapter{ОЦЕНИВАНИЕ НЕОПРЕДЕЛЕННОСТИ ИЗМЕРЕНИЙ...} \label{app_neopr}

%отмена выноса section в содержание
\settocdepth{chapter}


Согласно \cite{iso200898} процесс оценивания неопределенности можно представить в виде следующих этапов:

...

Результаты расчета представлены в виде таблицы \ref{table_uw}.




\begin{landscape}

\begin{table}[h]
\caption {Результаты расчёта неопределенности ...}
\label{table_uw}
\begin{center}
\begin{small}
\begin{spacing}{3}
\begin{tabular}{|c|c|c|c|c|c|c|c|c|c|c|c|c|}
\hline
{\textnumero ~слоя}
& {$u(m_\text{в+б})$, г}
& {$u(m_\text{c+б})$, г}
& {$u(m_\text{б})$, г}
& {$c_{m_\text{в+б}}$}
& {$c_{m_\text{с+б}}$}
& {$c_{m_\text{б}}$}
& {$u(w_s)$, \%}
& {$u(\delta w_v)$, \%}
& {$\overline{w}$, \%}
& {$u(w)$, \%}
& {$U(w)$, \%}
& {$w$, \%} \\
\hline


1&0,0115&0,0769&0,0115&0,918&-0,936&0,018&0,073&0,426&1,38&0,43&0,86& 1,38$\pm$0,86\\
\hline

2&0,0115&0,0775&0,0115&0,910&-0,957&0,047&0,075&0,127&3,67&0,15&0,29& 3,67$\pm$0,29\\
\hline

3&0,0115&0,0784&0,0115&0,901&-0,955&0,054&0,076&0,139&4,19&0,16&0,32& 4,19$\pm$0,32\\
\hline

4&0,0115&0,0797&0,0115&0,882&-0,929&0,047&0,075&0,130&3,78&0,15&0,30& 3,78$\pm$0,30\\
\hline

5&0,0115&0,0755&0,0115&0,940&-0,985&0,045&0,075&0,193&3,34&0,21&0,41& 3,34$\pm$0,41\\
\hline


\end{tabular}
\end{spacing}
\end{small}
\end{center}
\end{table}

\end{landscape}




\resettocdepth
\chapter{ЧТО-ТО ЕЩЁ}
\label{AppB}


В таблице \ref{tab_sorbc_gs} представлены значения ...




\begin{longtable}{|c|c|c|c|c|c|c|c|c|c|p{1.3cm}|}
\caption{Значения чего-то там при различной температуре}
\label{tab_sorbc_gs}
\\
\hline
Температура, & \multicolumn{10}{c|} {Некая величина, \%,}
\\
\textcelsius & \multicolumn{10}{c|} {при относительной влажности воздуха, \%}
\\
%\hline
{} & 10 &  33 &  40 &  55 &  75 &  80 &  85 &  90 &  97 & над водой\\
\hline
0 & - & - & - & - & 6,21 & - & - & - & - & -\\
\hline
2 & - & - & - & - & - & - & - & - & - & 178\\
\hline
1 & 0,20 & 1,00 & - & 10 & 15 & - & 20 & 50 & 100 & -\\
\hline
60 & - & - & - & - & 20 & - & - & - & - & -\\
\hline
-6 & - & - & 1,56 & - & - & 25 & - & - & - & -\\
\hline
-0 & - & - & - & - & - & 9,0 & - & - & 8,3 & -\\
\hline
\end{longtable}





\chapter{ИНФОРМАЦИЯ О ПРАКТИЧЕСКОМ ИСПОЛЬЗОВАНИИ РЕЗУЛЬТАТОВ РАБОТЫ} \label{app_vnedr}

Цветные сканы актов и справок о внедрениях с крупной надписью\\ <<КОПИЯ>>.


