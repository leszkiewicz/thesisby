
%\newpage
\chapter*{ОБЩАЯ ХАРАКТЕРИСТИКА РАБОТЫ} ~
\addcontentsline{toc}{chapter}{ОБЩАЯ ХАРАКТЕРИСТИКА РАБОТЫ}~

\vspace{-16pt}
%\subsubsection*{Связь работы с научными программами (проектами), темами.}
\textbf{Связь работы с научными программами (проектами), темами}

Тема работы соответствует программе <<Переливание из пустого в порожнее>> приоритетных направлений научно-технической деятельности в Республике Беларусь на \mbox{2016--2020}~годы, утвержденных Указом Президента Республики Беларусь  \textnumero ~100500 от~30.02.2055.


\textbf{Цель и задачи исследования}

\textit{Цель работы}
-- откорректировать переработанный шаблон диссертации и залить в нет для его возможности использования другими поколениями ну и просто <<каб не прапала>>.

\textit{Объект исследований} -- абракадабра.

%\vspace{-20pt}
%\subsubsection*{Задачи исследований:}~
\textit{Задачи исследования:}

--~задачи исследования можно подгрузить из отдельного файла (прямо этого);

--~и его же можно подгрузить где-нибудь в литобзоре когда ставится задачи исследований;

--~хотя позже мне посоветовали текст задач изменить, чтобы не было дословного копирования из литобзора;

--~хотя, кому оно мешает в двух местах -- я так и не понял;

--~спишем на бюрократию.
%
%\ifthenelse{\value{isthesis}=0}{}{\pagebreak[4]}

%
%\vspace{-20pt}
%\subsubsection*{Научная новизна}работы:
%\pagebreak[4]
\textbf{Научная новизна} работы:

--~получены ...;

--~усовершенствована ...;

--~разработана ... .



%\vspace{-20pt}
%\subsubsection*{Положения, выносимые на защиту:}~
\ifthenelse{\value{isthesis}=0}{\pagebreak[4]}{}
\textbf{Положения, выносимые на защиту:}

--~результаты ...;

%\ifthenelse{\value{isthesis}=0}{\pagebreak[4]}{}

--~способ ...;

--~метод.


%\vspace{-20pt}
%\subsubsection*{Личный вклад соискателя.}~
\textbf{Личный вклад соискателя}~

Работа выполнена автором в Белорусском национальном техническом университете под руководством ...

Исследования ... проведены совместно с ...

Исследования ... выполнены совместно с ...

Остальные исследования ..., разработка математических моделей, ... и расчёты выполнены автором самостоятельно. 


%\vspace{-20pt}
%\subsubsection*{Апробация диссертации и информация об использовании ее результатов.}~
%\ifthenelse{\value{isthesis}=1}{\pagebreak[4]}{}
\textbf{Апробация диссертации и информация об использовании \linebreak ее результатов}~

Основные положения и результаты работы докладывались и обсуждались в рамках:

--~Научно-технических конференций <<ХХ>> \textnumero ~1--1 (2016--2019~гг.),  \textnumero ~1--5 (2002--2007~гг.);

--~V Международной практической конференции <<ХХ>>, Москва, 13 мая 2012~г.;

--~Научном семинаре <<ХХ>>, Санкт-Петербург, 23--24 июля 2001~г.;

--~Научно-практической конференции <<ХХ>>, Минск, 20 августа 2021~г.


%\vspace{-20pt}
%\subsubsection*{Опубликование результатов диссертации.}~
\textbf{Опубликование результатов диссертации}

По результатам работы опубликовано:

--\,пять статей в научных изданиях, соответствующих перечню ВАК Республики Беларусь;

--\,четыре статьи в научных изданиях, соответствующих перечню ВАК при Минобрнауки России;

--\,две статьи в научно-технических изданиях;

--\,пять публикаций в сборниках материалов и тезисов по результатам научно-технических конференций.


%\vspace{-20pt}
%\subsubsection*{Структура и объём диссертации.}~
\ifthenelse{\value{isthesis}=0}{\pagebreak[4]}{}
\textbf{Структура и объём диссертации}


Работа состоит из введения, общей характеристики работы, основной части, заключения,
списка использованных источников и приложений.

Основная часть работы состоит из 5 глав.

В первой главе выполнен ...

Вторая глава ...


В третьей главе ...


В главе 4 ...

\ifthenelse{\value{isthesis}=1}{\pagebreak[4]}{}
В главе 5...


Объём диссертации составляет ХХ страницы, в том числе основная часть YY страниц.
Основная часть содержит ZZ рисунков и TT таблиц.
